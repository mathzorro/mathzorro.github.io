% Christopher Hanusa
% Christopher.Hanusa@qc.cuny.edu	
\documentclass[12pt]{article} 

\usepackage{amssymb,amsthm,amsmath,epsfig}

\usepackage{color}
\definecolor{darkblue}{rgb}{0, 0, .4}
\definecolor{grey}{rgb}{.7, .7, .7}

\setlength{\topmargin}{-.5in}
\setlength{\textheight}{9in}
\setlength{\textwidth}{6.5in}
\setlength{\oddsidemargin}{0in}
\setlength{\evensidemargin}{0in}

\newcommand{\lam}{\lambda}

\usepackage[breaklinks]{hyperref} 
\hypersetup{
    colorlinks=true,
    linkcolor=darkblue,
    anchorcolor=darkblue,
    citecolor=darkblue,
    pagecolor=darkblue,
    urlcolor=darkblue,
    pdftitle={},
    pdfauthor={}
}

\begin{document}
\pagestyle{empty}



\begin{center}\large 
MATH 245, Spring 2013

{\sc Homework 2}

due 10:45{\sc am} on Wednesday, February 19.
\end{center}

\noindent
{\em Background reading:} Section 1.3 to page 25 and Section 1.4.

\smallskip\noindent
Follow the posted homework guidelines when completing this assignment.  

\smallskip\noindent
You may not use any resources other than class material, your Math Modeling classmates, and your professor.  

\smallskip\noindent
Don't forget to include acknowledgments for those who helped you with the assignment!

\begin{enumerate}
\item[\bf 2-1.] (8 pts) Here is some data that represents an independent variable $x$ and a dependent variable $y$. 
\begin{center}
\begin{tabular}{|c|ccccccccccccc|}\hline
$x$ & 1 & 2 & 5 & 7 & 9 & 11 & 13 & 15 & 16 & 18 & 21 & 24 & 28\\ \hline
$y$ & 12 & 15 & 7 & 24 & 26 & 19 & 35 & 70 & 71 & 87 & 144 & 174 & 370 \\ \hline
\end{tabular}
\end{center}

It is thought that $y$ satisfies a function of type (a)~$y=Cx^k$ or type (b)~$y=Ck^x$, but it is not known which one is more likely.  
\begin{itemize}
\item Use the method of transforming the data using logarithms combined with {\bf visual fitting} to determine the curve of best fit.  (Use graph paper.)  \newline Do this twice---once for a curve of type (a) and once for a curve of type (b).
\item Now compare and contrast your two curves of best fit.  Create the residual graphs for each fit.  In a paragraph or two, discuss which one you think gives a better fit and why.  
\end{itemize}


\item[\bf 2-2.] (6 pts) Read \url{http://eagereyes.org/criticism/anscombes-quartet}.  This question is to understand the set of four figures in the middle of the page.  

Write three paragraphs explaining this blog post in the context of this class, addressing the following points.  
\begin{itemize}
\item Explain what makes the four data sets similar; why are they grouped together?  
\item Discuss the differences in the figures; does the ``line of best fit" fit one of the sets of data better than another?  
\item What does ``best fit" mean in this context?  
\item Last, given these four figures and the lines of best fit, how would you modify your modeling approach to find a better fit to each data set (if necessary).
\end{itemize}

\item[\bf 2-3.] 
(6 pts) Here is some data related to the growth of a plant after grafting:
\begin{center}
\begin{tabular}{|r|cccccc|}\hline
Months after grafting & 1 & 2 & 3 & 4 & 5 & 6 \\ \hline
Height, in inches & 0.8 & 2.4 & 4.0 & 5.1 & 7.3 & 9.4  \\ \hline
\end{tabular}
\end{center}

\begin{enumerate}
\item Do a {\bf linear} regression {\bf by hand} to determine the best fit line.   Work under the assumption that the height ($h$) is {\em proportional} to the time ($t$). 
\item Use the model from part (a) to predict the height of the graft at four and one-half months and again the height of the graft at 5 years.  Which prediction is more reliable?  Give specific reasons why one might be more reliable than the other.
\end{enumerate}

\end{enumerate}




\end{document}










