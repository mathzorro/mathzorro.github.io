% Christopher Hanusa
% Christopher.Hanusa@qc.cuny.edu	
\documentclass[12pt]{article} 

\usepackage{amssymb,amsthm,amsmath,epsfig}

\usepackage{color}
\definecolor{darkblue}{rgb}{0, 0, .4}
\definecolor{grey}{rgb}{.7, .7, .7}

\setlength{\topmargin}{-.5in}
\setlength{\textheight}{9in}
\setlength{\textwidth}{6.5in}
%\setlength{\headheight}{26pt}
%\setlength{\headsep}{2pt}
\setlength{\oddsidemargin}{0in}
\setlength{\evensidemargin}{0in}

\newcommand{\lam}{\lambda}

\usepackage[breaklinks]{hyperref} 
\hypersetup{
    colorlinks=true,
    linkcolor=darkblue,
    anchorcolor=darkblue,
    citecolor=darkblue,
    pagecolor=darkblue,
    urlcolor=darkblue,
    pdftitle={},
    pdfauthor={}
}

\begin{document}
\pagestyle{empty}

%%%%%%%%%%%%%%%%%%%%%%%%%%%%%%%%%%%%%%%%%%%
%%%%%%%%%%%%%%%%%%%%%%%%%%%%%%%%%%%%%%%%%%%
\begin{center}\large 
MATH 245, Spring 2014

{\sc Homework 1}

due 10:45{\sc am} on Monday, February 3.
\end{center}

\noindent
{\em Background reading:} Sections 1.1 and 1.2 to page 18.


\begin{enumerate}
\item[\bf 1-1.] Problem {\bf 1-1} must be completed online {\bf before class} on Wednesday 1/29 for credit.
\begin{enumerate}
\item Email me at \url{chanusa@qc.cuny.edu} with the following four things: (1) Your name, (2) Your class (Math 245), (3) the email address where you are best contacted, and (4) your expected graduation year. Thanks!
\item Thoroughly read the class web page including the syllabus and schedule. This should answer all the questions that you may have about the class.  Next, take the syllabus quiz on Blackboard.  {\bf Retake} the quiz as necessary to earn a score of 100\%. 
\end{enumerate}
\end{enumerate}
Problems {\bf 1-2} through {\bf 1-4} should be written up (or typed) and handed in as class starts on Monday 2/4:

\smallskip\noindent
{\color{red} Follow the posted homework guidelines when completing this assignment.}

\begin{enumerate}
\item[\bf 1-2.]  Read Section 1.1 from the textbook (pages 1--12); it gives six different models that describe real-life situations. Write two to three paragraphs explaining where you have experience with mathematical models in real life. (Do not use the examples from the book unless you feel you have a unique perspective.)  

In your discussion, be sure to explain how the real-life situation and the model differ. (You may want to address the following questions: Are there simplifying assumptions? Does the model truly describe what happens in real life or are there limits to the model's effectiveness? $\ldots$ )
\item[\bf 1-3.]  Below are three different vague scenarios. Choose {\bf one of the three} to focus on for this question.   Next, identify a {\bf precise problem statement} related to the scenario that you would like to study. Then, determine {\bf eight variables} that affect your proposed problem statement. Last, of those eight variables, choose those that are the most important, and {\bf explain why} you think they are more important than the rest.
\begin{itemize}
\item The MTA is considering implementing a fast bus corridor between Flushing and Jamaica.
\item A local TV station is determining its schedule for the next season.
\item Queens College would like to convince more students to play intramural sports.
\end{itemize}
\begin{center}
\bf (continued on next page...)
\end{center}
\newpage
\item[\bf 1-4.]  Here is some data that represents an independent variable $x$ and a dependent variable $y$.\!\!\!\!\!\!\!\!

\begin{center}
\begin{tabular}{|c|cccccccccc|}\hline
$x$ & 1 & 4 & 5 & 7 & 8 & 10 & 13 & 15 & 18 & 20 \\ \hline
$y$ & 4 & 4 & 3 & 8 & 9 & 8 & 10 & 14 & 17 & 13 \\ \hline
\end{tabular}
\end{center}
\begin{enumerate}
\item It is thought that $y$ satisfies a {\bf linear function} of $x$. Plot the data on a graph and give a rough estimate for this function. 
\item Explain in a few sentences how you found your answer.  
\item Do you expect your answer to be different if there is an assumption that $y$ is {\bf proportional} to $x$?
\end{enumerate}
\end{enumerate}


\end{document}










