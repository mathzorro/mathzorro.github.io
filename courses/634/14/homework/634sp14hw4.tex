% Chris Hanusa
% chanusa@qc.cuny.edu	
\documentclass[12pt]{article} 

\usepackage{amssymb,amsthm,amsmath,epsfig}

\usepackage{color}
\definecolor{darkblue}{rgb}{0, 0, .4}
\definecolor{grey}{rgb}{.7, .7, .7}

\setlength{\topmargin}{-.5in}
\setlength{\textheight}{9in}
\setlength{\textwidth}{6.5in}
%\setlength{\headheight}{26pt}
%\setlength{\headsep}{2pt}
\setlength{\oddsidemargin}{0in}
\setlength{\evensidemargin}{0in}

\newcommand{\lam}{\lambda}

\usepackage[breaklinks]{hyperref} 
\hypersetup{
    colorlinks=true,
    linkcolor=darkblue,
    anchorcolor=darkblue,
    citecolor=darkblue,
    pagecolor=darkblue,
    urlcolor=darkblue,
    pdftitle={},
    pdfauthor={}
}

\begin{document}
\pagestyle{empty}

%%%%%%%%%%%%%%%%%%%%%%%%%%%%%%%%%%%%%%%%%%%%%%%%%%%%%%%%%%%%%%%%%%%%
%%%%%%%%%%%%%%%%%%%%%%%%%%%%%%%%%%%%%%%%%%%%%%%%%%%%%%%%%%%%%%%%%%%%
\begin{center}\large 
MATH 634, Spring 2014

{\sc Homework 4}

to be prepared for presentation at  5:00{\sc pm} on Monday, February 10.
\end{center}

\noindent
{\em Background reading:} {\em Pearls in Graph Theory}, Sections 1.1 through 1.3.


\begin{enumerate}
\item[\bf 4-1.] 
%12h1, 13h1 
%Pearls 1.1.2ab
In parts (a) and (b) below, do not apply Theorem 1.1.2.  
[{\em Hint: You will need to find two families of graphs that give an answer for every possible value of $n$.}]
\begin{itemize}
\item Prove that for every even number $n\geq 4$, there exists a graph with $n$ vertices, all of which have degree $3$.
\item Prove that for every odd number $n\geq 5$, there exists a graph with $n+1$ vertices, $n$ of which have degree $3$.
\end{itemize}  


\item[\bf 4-2.]
%09h2,12h2,13h2
% Pearls 1.1.7
Prove that no graph has all degrees different.  That is, prove that in a degree sequence of a graph, there is at least one repeated number.

\item[\bf 4-3.]
%09h3,11h1,13h2
Explore the proof of Theorem 1.1.2.

The graph below has degree sequence ($\mathcal S_1$) $4\,4\, 3\, 3\, 3\, 2\, 2\, 2\, 2\, 1$.  Define ($\mathcal S_2$) to be $3\, 2\, 2\, 2\, 2\, 2\, 2\, 2\, 1$. Walk through the steps of the proof of Theorem 1.1.2 in the following way.

First, let us choose vertex $c$ from the graph to be vertex $S$ from the proof. Next, assign to each of the remaining vertices ($a$ -- $j$) a name of the form $T_i$ or $D_i$, just as in the proof.
\begin{enumerate}
\item If you delete vertex $S$, does the new graph have degree sequence ($\mathcal S_2$)?
\item Use the method in the proof to modify the original graph (possibly applying the algorithm multiple times) so that the resulting graph is such that removing $S$ gives a graph with degree sequence ($\mathcal S_2$).
\end{enumerate}

\begin{center}
\includegraphics[height=2in]{figures/G1.eps}
\end{center}

\item[\bf 4-4.]
%11h2,13h3
\begin{enumerate}
\item Prove that if $n$ is large enough, then the following statement is true:  
\begin{quote}For all graphs on $n$ vertices, either $G$ or $G^c$ contains a cycle.
\end{quote}
\item For which $n$ does this start to be true?
\end{enumerate}


\item[\bf 4-5.] Let $G$ be a graph with $n$ vertices and $n$ edges.
%09h4,12p,13h3
\begin{enumerate}
\item Suppose $G$ is connected.  How many cycles does $G$ have?  Prove it.  
\item Suppose $G$ is {\bf NOT} connected.  What can you say about the number of cycles in the graph?  Can you determine a formula? 
\end{enumerate}
[{\em Part \textup{(b)} is an exploration question.  I want you to explore what happens and write up as much as you can about what you learn.}]

\end{enumerate}

\end{document}














