% Chris Hanusa
% chanusa@qc.cuny.edu	

\documentclass[12pt]{article} 

\usepackage{amssymb,amsthm,amsmath,epsfig}

\usepackage{color}
\definecolor{darkblue}{rgb}{0, 0, .4}
\definecolor{grey}{rgb}{.7, .7, .7}

\setlength{\topmargin}{-.5in}
\setlength{\textheight}{9in}
\setlength{\textwidth}{6.5in}
\setlength{\oddsidemargin}{0in}
\setlength{\evensidemargin}{0in}

\newcommand{\lam}{\lambda}

\usepackage[breaklinks]{hyperref} 
\hypersetup{
    colorlinks=true,
    linkcolor=darkblue,
    anchorcolor=darkblue,
    citecolor=darkblue,
    pagecolor=darkblue,
    urlcolor=darkblue,
    pdftitle={},
    pdfauthor={}
}

\begin{document}
\pagestyle{empty}

%%%%%%%%%%%%%%%%%%%%%%%%%%%%%%%%%%%%%%%%%%%%%%%%%%%%%%%%%%%%%%%%%%%%
%%%%%%%%%%%%%%%%%%%%%%%%%%%%%%%%%%%%%%%%%%%%%%%%%%%%%%%%%%%%%%%%%%%%
\begin{center}\large 
MATH 634, Spring 2014

{\sc Homework 1}

due 4:30{\sc pm} on Wednesday, January 29.
\end{center}

\noindent
{\em Background reading:} {\em Pearls in Graph Theory}, Section 1.1.

\smallskip\noindent
{\color{red} Follow the posted homework guidelines when completing this assignment.}

\smallskip\noindent
Problems {\bf 1D}, {\bf 1P}, and {\bf 1E} should be typed (or written up) and handed in as class starts on Wednesday 1/29:


\begin{enumerate}
\item[\bf 1D.]  ({\bf D} stands for Definitions.)

\begin{itemize}
\item simple graph
\item graphic sequence
\item bipartite graph
\item complete bipartite graph $K_{m,n}$
\item wheel graph $W_n$
\end{itemize}

{\em Reminder:} For each of these vocabulary words, 
\begin{enumerate}
\item Give a precise definition statement for the vocabulary word.
\item Explain in your own words what you understand this precise definition to mean. 
\item Give one or more examples that highlight the vocabulary word.

  (It may make sense to give two examples: a true example and a non-example.)
\end{enumerate}

\item[\bf 1E.]  ({\bf E} stands for Exploration.)

Write a paragraph or two giving an example of where you have seen graphs in real life. (Do not use the examples from class unless you have a unique perspective.)

 For the example you give, 
\begin{enumerate}
\item {\bf Explain} what corresponds to the abstract concepts of 
\begin{itemize}
\item vertices
\item edges
\item vertex degree
\end{itemize}
\item {\bf Discuss} whether a vertex can have a degree of zero or one.
\end{enumerate}

\item[\bf 1P.]  ({\bf P} stands for Proof.)

Seven students go on vacations.  They decide that each will send a postcard to three of the others.  Is it possible that every student receives postcards from precisely the three to whom he sends postcards? \newline [{\em Note: If you intend to use graph theory, explain why your reasoning applies.}]

\end{enumerate}

\bigskip\noindent
{\em Reminder:} For {\bf Proof} questions, you may not use any resources other than class material, your classmates, and your professor.  In particular, no internet searches may be used to solve this problem.  On the other hand, it is permitted and sometimes encouraged to consult the internet or additional resources for help in understanding the {\bf Definition} questions or completing the {\bf Exploration} questions.




\end{document}














